%%%%%%%%%%%%%%%%%%%%%%%%%%%%%%%%%%%%%%%%%
% University/School Laboratory Report
% LaTeX Template
% Version 3.1 (25/3/14)
%
% This template has been downloaded from:
% http://www.LaTeXTemplates.com
%
% Original author:
% Linux and Unix Users Group at Virginia Tech Wiki 
% (https://vtluug.org/wiki/Example_LaTeX_chem_lab_report)
%
% License:
% CC BY-NC-SA 3.0 (http://creativecommons.org/licenses/by-nc-sa/3.0/)
%
%%%%%%%%%%%%%%%%%%%%%%%%%%%%%%%%%%%%%%%%%

%----------------------------------------------------------------------------------------
%	PACKAGES AND DOCUMENT CONFIGURATIONS
%----------------------------------------------------------------------------------------

\documentclass{article}

\usepackage{graphicx} % Required for the inclusion of images
\usepackage{natbib} % Required to change bibliography style to APA
\usepackage{acronym} 
\usepackage{hyperref}

%% Code blocks

\usepackage{listings} % Required for inserting code snippets
\usepackage[usenames,dvipsnames]{color} % Required for specifying custom colors and referring to colors by name

\definecolor{DarkGreen}{rgb}{0.0,0.4,0.0} % Comment color
\definecolor{highlight}{RGB}{255,251,204} % Code highlight color

\lstdefinestyle{Style1}{ % Define a style for your code snippet, multiple definitions can be made if, for example, you wish to insert multiple code snippets using different programming languages into one document
language=Bash, % Detects keywords, comments, strings, functions, etc for the language specified
backgroundcolor=\color{highlight}, % Set the background color for the snippet - useful for highlighting
basicstyle=\footnotesize\ttfamily, % The default font size and style of the code
breakatwhitespace=false, % If true, only allows line breaks at white space
breaklines=true, % Automatic line breaking (prevents code from protruding outside the box)
captionpos=b, % Sets the caption position: b for bottom; t for top
commentstyle=\usefont{T1}{pcr}{m}{sl}\color{DarkGreen}, % Style of comments within the code - dark green courier font
deletekeywords={}, % If you want to delete any keywords from the current language separate them by commas
%escapeinside={\%}, % This allows you to escape to LaTeX using the character in the bracket
firstnumber=1, % Line numbers begin at line 1
frame=single, % Frame around the code box, value can be: none, leftline, topline, bottomline, lines, single, shadowbox
frameround=tttt, % Rounds the corners of the frame for the top left, top right, bottom left and bottom right positions
keywordstyle=\color{Blue}\bf, % Functions are bold and blue
morekeywords={}, % Add any functions no included by default here separated by commas
numbers=left, % Location of line numbers, can take the values of: none, left, right
numbersep=10pt, % Distance of line numbers from the code box
numberstyle=\tiny\color{Gray}, % Style used for line numbers
rulecolor=\color{black}, % Frame border color
showstringspaces=false, % Don't put marks in string spaces
showtabs=false, % Display tabs in the code as lines
stepnumber=5, % The step distance between line numbers, i.e. how often will lines be numbered
stringstyle=\color{Purple}, % Strings are purple
tabsize=2, % Number of spaces per tab in the code
}

% Create a command to cleanly insert a snippet with the style above anywhere in the document
\newcommand{\insertcode}[2]{\begin{itemize}\item[]\lstinputlisting[caption=#2,label=#1,style=Style1]{#1}\end{itemize}} % The first argument is the script location/filename and the second is a caption for the listing

\setlength\parindent{0pt} % Removes all indentation from paragraphs

\renewcommand{\labelenumi}{\alph{enumi}.} % Make numbering in the enumerate environment by letter rather than number (e.g. section 6)

%\usepackage{times} % Uncomment to use the Times New Roman font

%----------------------------------------------------------------------------------------
%	DOCUMENT INFORMATION
%----------------------------------------------------------------------------------------

\title{Deploying the CSDT Community Site} % Title

\author{GK-12 @ RPI} % Author name

\date{\today} % Date for the report

% Acronyms

\acrodef{ccs}[CCS]{CSDT Community Site}

\begin{document}

\maketitle % Insert the title, author and date

% If you wish to include an abstract, uncomment the lines below
% \begin{abstract}
% Abstract text
% \end{abstract}

%----------------------------------------------------------------------------------------
%	SECTION 1
%----------------------------------------------------------------------------------------

\section{Objective}

To document the procedure for deploying an updated version of the \ac{ccs} server package. This document is intended for developers of the CCS or CSnap.


% If you have more than one objective, uncomment the below:
%\begin{description}
%\item[First Objective] \hfill \\
%Objective 1 text
%\item[Second Objective] \hfill \\
%Objective 2 text
%\end{description}

\subsection{Definitions}
\label{definitions}
\begin{description}
\item[\acf{ccs}]
The \acl{ccs} software is a Django-based application which allows users to easily share CSnap projects, ideas, and other material related to the GK-12 grant at RPI.
For more information, please see the document (Insert bilbo reference here)
\item[CSnap]
CSnap is a drag-and-drop visual programming environment. 
It integrates with the \ac{ccs} to help students learn programming skills.
For more information, please see the document (Insert bilbo reference here)
\end{description}

\subsection{Big Picture}

The procedure works as follows:
\begin{enumerate}
 \item Schedule downtime, bring down the live server
 \item Clone the server hosting the live \ac{ccs}
 \item Perform updates on the cloned server
 \item Run all unit tests
 \item Run all manual tests
 \item Bring up the live server
\end{enumerate}

\section{Clone the live server}

The live server contains \ac{ccs} application, live database, and uploaded user files.
Before performing an upgrade, we will clone the live server so if something goes wrong, we don't lose any data.

\subsection{Procedure}

Schedule time with \href{mailto:COPPEE@rpi.edu}{Ethan Coppenrath (\nolinkurl{COPPEE@rpi.edu})} to update the community site.
During this time, Ethan will need to:
\begin{enumerate}
 \item Disconnect the live server from the internet
 \item Clone the live server
 \item Give the developer time to update the cloned server
 \item Connect the cloned server to the internet as the live server
\end{enumerate}

If something goes wrong and results in a longer than anticipated down time (more than an hour), the old live server should be brought back online while the developer debugs the changes.

\section{Updating on the cloned server}

The first step in updating is to gain access to the server.
After gaining access, update the application, static files, and migrate the database.

\subsection{Gaining access}

Login information is stored seperately in the csdt\_doc\_private respository, which the developer will need access to.
To gain access to this repository, contact \href{mailto:eglash@rpi.edu}{Ron Eglash (\nolinkurl{eglash@rpi.edu})}.

\subsection{Updating the application}

This section depends a lot on what changes were made.
In general, it will involve:
\begin{enumerate}
 \item Performing a git-pull from the master repository on Github
 \item Updating all installed Django/Python modules
 \item Performing updates for any linked projects (such as CSnap)
 \item Updating the collected static files
 \item Migrating the database
\end{enumerate}

\subsubsection{Updating \ac{ccs}}

Run the commands in listing \ref{scripts/update-ccs.txt} one at a time, and deal with errors as they appear.
Expected output is displayed below each command.
Large outputs are truncated.

\label{update-ccs}
\insertcode{scripts/update-ccs.txt}{Updating the application}

\section{Running all unit tests}

Run the commands in listing \ref{scripts/unitTest-css.txt} to perform the built-in unit tests.

\label{unitTest-css}
\insertcode{scripts/unitTest-css.txt}{Running unit tests}

If the unit test reports any failures, fix the \ac{ccs} so that all tests pass before continuing.

\section{Run manual tests}

Connect to the server by adjusting the your configuration so that \url{http://community.csdt.rpi.edu} points to the IP of the server you are configuring.

Run the tests documented in (insert reference to the manual testing document here).

\section{Bringing up the live site}

Once all updates and tests have been completed, bring the site back up.
To do this, simply update the IP on the cloned machine so that it replaces the old machine.

%----------------------------------------------------------------------------------------
%	BIBLIOGRAPHY
%----------------------------------------------------------------------------------------

\bibliographystyle{apalike}

\bibliography{sample}

%----------------------------------------------------------------------------------------


\end{document}